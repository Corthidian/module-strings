\documentclass[12pt]{scrartcl}


\setlength{\parindent}{0pt}
\setlength{\parskip}{.25cm}

\usepackage{graphicx}

\usepackage{xcolor}

\definecolor{darkred}{rgb}{0.5,0,0}
\definecolor{darkgreen}{rgb}{0,0.5,0}
\usepackage{hyperref}
\hypersetup{
  letterpaper,
  colorlinks,
  linkcolor=red,
  citecolor=darkgreen,
  menucolor=darkred,
  urlcolor=blue,
  pdfpagemode=none,
  pdftitle={CSCE 155E - Lab 8.0 - Strings},
  pdfauthor={Christopher M. Bourke},
  pdfkeywords={}
}

\definecolor{MyDarkBlue}{rgb}{0,0.08,0.45}
\definecolor{MyDarkRed}{rgb}{0.45,0.08,0}
\definecolor{MyDarkGreen}{rgb}{0.08,0.45,0.08}

\definecolor{mintedBackground}{rgb}{0.95,0.95,0.95}
\definecolor{mintedInlineBackground}{rgb}{.90,.90,1}

%\usepackage{newfloat}
\usepackage[newfloat=true]{minted}
\setminted{mathescape,
               linenos,
               autogobble,
               frame=none,
               framesep=2mm,
               framerule=0.4pt,
               %label=foo,
               xleftmargin=2em,
               xrightmargin=0em,
               startinline=true,  %PHP only, allow it to omit the PHP Tags *** with this option, variables using dollar sign in comments are treated as latex math
               numbersep=10pt, %gap between line numbers and start of line
               style=default, %syntax highlighting style, default is "default"
               			    %gallery: http://help.farbox.com/pygments.html
			    	    %list available: pygmentize -L styles
               bgcolor=mintedBackground} %prevents breaking across pages
               
\setmintedinline{bgcolor={mintedBackground}}
\setminted[text]{bgcolor={mintedBackground},linenos=false,autogobble,xleftmargin=1em}
%\setminted[php]{bgcolor=mintedBackgroundPHP} %startinline=True}
\SetupFloatingEnvironment{listing}{name=Code Sample}
\SetupFloatingEnvironment{listing}{listname=List of Code Samples}


\title{CSCE 155 -- C}
\subtitle{Lab 8.0 -- Strings}
\author{~}
\date{~}

\begin{document}

\maketitle

\section*{Prior to Lab}

\section*{Prior to Lab}

Before attending this lab:
\begin{enumerate}
  \item Read and familiarize yourself with this handout.
  \item Read Chapters 8 and 21 of the \href{http://cse.unl.edu/~cbourke/ComputerScienceOne.pdf}{Computer Science I} textbook
  \item Watch Videos 8.1 thru 8.3 of the \href{https://www.youtube.com/playlist?list=PL4IH6CVPpTZVkiEnCEOdGbYsFEdtKc5Bx}{Computer Science I} video series
\end{enumerate}

\section*{Peer Programming Pair-Up}

\textbf{For students in the online section:} you may complete
the lab on your own if you wish or you may team up with a partner
of your choosing, or, you may consult with a lab instructor to get
teamed up online (via Zoom).

\textbf{For students in the face-to-face section:} your
lab instructor will team you up with a partner.  

To encourage collaboration and a team environment, labs are be
structured in a \emph{peer programming} setup.  At the start of
each lab, you will be randomly paired up with another student 
(conflicts such as absences will be dealt with by the lab instructor).
One of you will be designated the \emph{driver} and the other
the \emph{navigator}.  

The navigator will be responsible for reading the instructions and
telling the driver what to do next.  The driver will be in charge of the
keyboard and workstation.  Both driver and navigator are responsible
for suggesting fixes and solutions together.  Neither the navigator
nor the driver is ``in charge.''  Beyond your immediate pairing, you
are encouraged to help and interact and with other pairs in the lab.

Each week you should alternate: if you were a driver last week, 
be a navigator next, etc.  Resolve any issues (you were both drivers
last week) within your pair.  Ask the lab instructor to resolve issues
only when you cannot come to a consensus.  

Because of the peer programming setup of labs, it is absolutely 
essential that you complete any pre-lab activities and familiarize
yourself with the handouts prior to coming to lab.  Failure to do
so will negatively impact your ability to collaborate and work with 
others which may mean that you will not be able to complete the
lab.  

\section{Lab Objectives \& Topics}
At the end of this lab you should be familiar with the following
\begin{itemize}
  \item Declare and print a string in C
  \item Manipulate strings in a variety of ways
  \item Use some basic functions from the \mintinline{text}{string.h} library
\end{itemize}

\section{Background}

Strings are collections of characters.  In C, Strings are represented using 
arrays of char values terminated by a special null-terminating character, 
\mintinline{text}{\0}.  Because they are arrays, the same precautions must 
be made when working with strings as with general arrays.  The standard 
string library (\mintinline{text}{string.h}) provides many convenience 
functions to manipulate and use strings.

This lab will familiarize you with some of these functions.  In particular, 
you will complete a program that implements a common children's game, 
horse (also known as hangman).  In this game, an English word is chosen 
at random and its characters hidden.  The player takes turns by guessing 
a letter; each instance (if any) of the guessed letter is revealed.  If the 
user is able to guess the word before a certain number of guessed letters 
then they win.  If they run out of guesses then they lose.

Most of the game mechanics have been implemented for you.  However, 
you will need to complete the game by implementing several functions 
used by the game to manipulate and compare strings.

\section{Activities}

Clone the code for this lab from GitHub using the following URL: 
\url{https://github.com/cbourke/CS1-C-Strings}.

\subsection{Implementing String Manipulation Functions}

\begin{enumerate}
  \item Navigate to the \mintinline{text}{src} directory and open 
  	\mintinline{text}{gameFunctions.c} in the editor of your choice.
  \item There are several functions already fully implemented in this file.  
	Your task for this lab is to implement the following four functions:
	\begin{itemize}
	  \item \mintinline{c}{initializeBlankString()} - this function should
	  take two arguments and return nothing:
	  \begin{itemize}
	    \item The first argument is an integer denoting the length of a string
	    \item The second argument should be a string
	    \item The function should alter the passed string and set all of
	    its characters to the underscore, \mintinline{c}{_} and properly
	    null-terminate it
	  \end{itemize}
	  \item \mintinline{c}{printWithSpaces()} - this function should take
	  one argument and return nothing:
	  \begin{itemize}
	    \item The argument should be a string
	    \item The function should print the contents of the string 
	    character-by-character with spaces between each one.
	    \item Hint: use the \mintinline{c}{strlen()} function to find 
	  the length of the passed string)
	  \end{itemize}
	  \item \mintinline{c}{revealGuessedLetter()} - this function should
	  take three arguments and return an integer:
	  \begin{itemize}
	    \item The first argument will be the solution string (that should
	    not be altered; use the \mintinline{c}{const} modifier to ensure
	    that it is not)
	    \item The second argument will be the (partially) revealed string
	    that you will alter
	    \item The third argument will be a single \mintinline{c}{char}, 
	    the letter guessed by the user in the game
	    \item Iterate over each character in the second argument and ``reveal''
	    the letter if it matches the guessed letter 
	    \item For example, if the first string is \mintinline{c}{"dinosaur"} 
	    and the second is \mintinline{c}{"________"} and the character passed 
	    is \mintinline{c}{a}, then the function should alter the second string 
	    so that it becomes \mintinline{c}{"_____a__"}.
	    \item You may assume that the strings are of equal length.   
	    \item The function should return a 1 if any letters were changed 
	    in the second string and 0 otherwise.
	  \end{itemize}
	  \item \mintinline{c}{checkGuess()} - this function should take 
	  two strings as input and return an integer.
	  \begin{itemize}
	    \item If the two strings are equivalent, return a 1 from the 
	    function.  If they are different, return a 0.  
	    \item Hint: make use the \mintinline{c}{strcmp()} function from 
	    the string library.
	  \end{itemize}
	\end{itemize}

  \item Navigate to the \mintinline{text}{include} directory and open 
	\mintinline{text}{gameFunctions.h} in the editor of your choice
  \item Complete the function prototypes that you implemented in 
	\mintinline{text}{gameFunctions.c} here.  
  \item Compile the program using the \mintinline{text}{make} command and 
	complete the worksheet.  
\end{enumerate}

\section{Handin/Grader Instructions}

\begin{enumerate}
  \item Hand in your completed files:
  \begin{itemize}
    \item \mintinline{text}{gameFunctions.c}
    \item \mintinline{text}{gameFunctions.h}
    \item \mintinline{text}{main.c}
    \item \mintinline{text}{worksheet.md}
  \end{itemize}
  through the webhandin (\url{https://cse-apps.unl.edu/handin}) 
  using your cse login and password.  
  \item Even if you worked with a partner, you \emph{both} should
  turn in all files.
  \item Verify your program by grading yourself through the
  webgrader (\url{https://cse.unl.edu/~cse155e/grade/}) using the
  same credentials.
  \item For this lab the grader will simulate playing 
  several games (some intentionally losing, some winning).
  Each run may be slightly different but the output should
  correctly display the game being played.
\end{enumerate}

\section{Advanced Activity (Optional)}

Currently, the game has a strict limitation on the number and length of 
words a user can enter in the \mintinline{text}{dictionary.txt} file.  
Alter the program so that it can accept any number of words and words 
of any length from \mintinline{text}{dictionary.txt} (hint: you'll 
need to dynamically allocate the memory for the array in 
\mintinline{text}{main.c}, among other changes in 
\mintinline{text}{gameFunctions.h} and \mintinline{text}{gameFunctions.c}).  

\end{document}
